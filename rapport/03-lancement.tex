\section{Comment lancer le projet?}


\subsection{Creation du fichier .jar }
\begin{enumerate}
    \item Étape 1 : Créer un projet Java dans Eclipse

        \begin{itemize}
            \item Ouvrez Eclipse et créez un nouveau projet Java ou ouvrez un projet Java existant dans lequel vous souhaitez créer un fichier .jar exécutable.
        \end{itemize}

\item Étape 2 : Configurer le fichier manifeste
\begin{itemize}
    \item Dans le projet Eclipse, créez un dossier nommé "META-INF" à la racine du projet, s'il n'existe pas déjà.
   \item À l'intérieur du dossier "META-INF", créez un fichier appelé "MANIFEST.MF" (assurez-vous de respecter la casse dans le nom du fichier et de l'enregistrer en tant que fichier texte).
    \item Ouvrez le fichier "MANIFEST.MF" dans un éditeur de texte.
    \item Ajoutez les informations nécessaires au manifeste. Par exemple, vous pouvez inclure les lignes suivantes :
 makefile

\fbox{
  
    \begin{minipage}{0.4\textwidth}
    \centering
        Manifest-Version: 1.0
        Main-Class: package.MainClass
    \end{minipage}
}




\item Assurez-vous de remplacer "package.MainClass" par le nom de la classe principale de votre application Java.
\end{itemize}
\item Étape 3 : Exporter le fichier .jar

\begin{itemize}
    \item Cliquez avec le bouton droit sur le projet dans la vue "Package Explorer" ou "Project Explorer" d'Eclipse.
    \item Sélectionnez "Export" dans le menu contextuel.
    \item Dans la boîte de dialogue "Export", développez la catégorie "Java" et sélectionnez "Runnable JAR file".
    \item Cliquez sur le bouton "Next".
    \item Sélectionnez le projet et la classe principale à utiliser comme point d'entrée de l'application.
    \item  Choisissez un emplacement et un nom de fichier pour le fichier .jar exporté.
    \item Sélectionnez les options d'exportation appropriées (telles que l'exportation des bibliothèques externes dans le fichier .jar ou non).
    \item  Cliquez sur le bouton "Finish" pour exporter le fichier .jar.
\end{itemize}
  
\end{enumerate}

\subsection{Lancement du projet}
\begin{enumerate}
    \item \label{lancer} On se position dans le dossier l-systems-renderer/dist
    \item On lance le terminal
    \item On tape la commande \textbf{java -jar run.jar}
\end{enumerate}



